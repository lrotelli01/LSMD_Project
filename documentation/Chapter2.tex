\chapter{Design}
\section{Requirements}
\subsection{Functional requirements}
\subsection{Guest}
\begin{itemize}
    \item The system must allow guests to register on the platform.
\end{itemize}

\subsection{Customer}
\begin{itemize}
    \item The system must allow customers to log in.
    \item The system must allow customers to log out.
    \item The system must allow customers to delete their account.
    \item The system must allow customers to book one or more rooms.
    \item The system must provide a messaging system to communicate with the property manager.
    \item The system must allow customers to publish their own reviews.
    \item The system must allow customers to modify their own reviews.
    \item The system must allow customers to delete their own reviews.
    \item The system must allow customers to add their payment method
    \item The system must allow customers to remove their payment method
    \item The system must allow customers to see their payment method
    \item The system must allow customers to cancel one or more existing bookings.
    \item The system must allow customers to modify one or more existing bookings.
    \item The system must allow customers to mark specific properties as favorites for future reference.
   \item The system must allow customers to remove specific properties as     favorites.
    \item The system must allow customers to view their booking history.
    \item The system must notify the manager when a customer/manager sends a message through the messaging system.
    \item The system must allow customers to create a pending booking while awaiting payment confirmation.
    \item The system must allow customers to delete a pending booking.
    \item The system must notify the customer in case of:
    \begin{itemize}
        \item booking confirmation or modification;
        \item booking cancellation.
    \end{itemize}
\end{itemize}

\subsection{Customer and Guest}
\begin{itemize}
    \item The system must allow guests to search and view properties managed by hosts.
    \item The system must allow guests to view a map showing the geographical distribution of properties.
    \item The system must allow users to view reviews written by other users.
    \item The system must allow users to filter properties by category.
    \item The system must allow users to order properties based on star rating.
    \item The system must allow users to filter properties and rooms by price range.
    \item The system must allow users to filter rooms by available services.
    \item The system must provide a view showing rooms offering a specific service.
    \item The system must provide a view showing properties offering a specific service.
    \item The system must allow users to view photos related to properties.
    \item The system must allow users to view detailed room information.
    \item The system must display room availability in real time.
    \item The system must display a list of Points of Interest (POIs) located in the proximity of the property on the property detail page.
    \item The system must provide personalized recommendations by displaying properties booked by other users who also reserved the property currently being viewed.
    \item The system must display the number of concurrent users currently viewing the same room to the user.
    \item The system must display the top 10 most visited properties within the last hour.
    \item The system must display a list of recently viewed rooms to the user.
    \item The system must recommend alternative properties that share the highest number of amenities with the property currently being viewed.


\end{itemize}

\subsection{Manager}
\begin{itemize}
    \item The system must allow the manager to add a property.
    \item The system must allow the manager to delete a property.
    \item The system must allow the manager to modify property information.
    \item The system must allow the manager to add a room.
    \item The system must allow the manager to delete a room.
    \item The system must allow the manager to modify room information.
    \item The system must allow the manager to view analytics by period and by property.
    \item The system must allow the manager to view properties information.
    \item The system must allow the manager to view analytics starting from the moment a property is added.
    \item The system must notify the manager with a notification in case of a new message from a customer
    \item The system must notify the manager with a notification in case of a new reservation from a customer
    \item The system must notify the manager with a notification if a reservation is removed by a customer.
    \item The system must notify the manager with a notification if a reservation is modified by a customer
    \item The system must allow the manager to view the payment status of rooms.
    \item The system must allow the manager to reply to user reviews.
    \item The system must allow the Manager to see all bookings for his properties
    \item The system must display the room occupancy rate within the analytics dashboard for a specific time period.
    \item The system must display the overall review rating aggregated over a selected time period.
    \item The system must show the total number of bookings made during a specific time period.
    \item The system must display the total count of reviews received over a selected time period.
    \item The system must identify and display the most frequently booked room type for a given time period.
    \item The system must calculate the average number of guests per room per booking for a specific time period.
    \item The system must display the total revenue generated over a selected time period.
    \item The system must list the reviews received during a specific time period.

\end{itemize}

\subsubsection{Admin}
\begin{itemize}
    \item The system must allow the admin to delete any review.
    \item The system must allow the admin to delete any property.
    \item The system must allow the admin to delete any room.
    \item The system must allow the admin to ban any user.
    \item The system must allow the admin to view analytics of any property.
    \item The system must allow the admin to filter analytics by time period for any property.
    \item The system must allow the admin to view analytics starting from the moment a property is added.
\end{itemize}

\subsection{Non-Functional Requirements}

\begin{itemize}
    \item Bookings must have a duration of one day.
    \item Passwords of registered users must be securely encrypted.
    \item Uploaded images must be in JPG or JPEG format.
    \item The system must provide notifications via email or application.
    \item The application navigation must be user-friendly.
    \item The results produced by the system must be consistent.
    \item The system must ensure high data availability.
    \item The system must ensune partition tolerance.
    \item The system must manage data storage using Document DB, Graph DB, and Key-Value DB.
    \item The DBMS technologies used must be Neo4j, MongoDB, and Redis.
    \item The business logic and data access layers must be implemented using the Spring framework in Java.
    \item The codebase must be easily maintainable, readable, and updatable.
    \item The system must use a document database to store information permanently.
    \item The system must use a key-value database to temporarily store information, manage cache, locks, and user sessions.
    \item The system must implement a stateless authentication mechanism using JSON Web Tokens (JWT). All protected API resources must require a valid Bearer token in the Authorization header to grant access.
\end{itemize}

\subsection{Use Cases Diagram}

\begin{figure}[H]
    \centering
    \includegraphics[width=1\textwidth]{images/UseCasesDiagram.png}
    \caption{Use cases diagram}
    \label{fig:comparison}
\end{figure}

\subsection{Analysis Class Diagram}

\begin{figure}[H]
    \centering
    \includegraphics[width=1\textwidth]{images/ClassAnalysisDiagram.png}
    \caption{Analysis Class diagram}
    \label{fig:comparison}
\end{figure}

\subsection{Mock-ups}
\subsection{Guest and customer view}
\begin{figure}[H]
    \centering
    \includegraphics[width=1\textwidth]{images/ShowProperties.png}
    \caption{Showing properties view}
    \label{fig:comparison}
\end{figure}
\begin{figure}[H]
    \centering
    \includegraphics[width=1\textwidth]{images/ShowRooms.png}
    \caption{Showing rooms view}
    \label{fig:comparison}
\end{figure}
\begin{figure}[H]
    \centering
    \includegraphics[width=1\textwidth]{images/RoomDetails.png}
    \caption{Room details view}
    \label{fig:comparison}
\end{figure}
\subsubsection{Customer}
\begin{figure}[H]
    \centering
    \includegraphics[width=1\textwidth]{images/CustomerBookings.png}
    \caption{Customer bookings view}
    \label{fig:comparison}
\end{figure}
\subsubsection{Manager view}
\begin{figure}[H]
    \centering
    \includegraphics[width=1\textwidth]{images/Analytics.png}
    \caption{Analytics properties view}
    \label{fig:comparison}
\end{figure}
\begin{figure}[H]
    \centering
    \includegraphics[width=1\textwidth]{images/ManagePropertiesManager.png}
    \caption{Manager properties view}
    \label{fig:comparison}
\end{figure}

\section{CAP Theorem and Consistency Management}
\begin{figure}[H]
    \centering
    \includegraphics[width=1\textwidth]{images/CapTheorem.png}
    \caption{Cap Theorem}
    \label{fig:comparison}
\end{figure}
In the design of the distributed architecture for the B\&B management platform, we addressed the trade-offs imposed by the \textbf{CAP Theorem} (Consistency, Availability, and Partition Tolerance). Given that network partitions (P) are unavoidable in a distributed environment involving microservices and cloud databases, the system adopts a hybrid strategy depending on the criticality of the specific business domain.

\subsection{Reservation Service: Prioritizing Consistency (CP)}
For the critical path of booking and payments, the system prioritizes \textbf{Consistency} over Availability. The business logic dictates that avoiding "double bookings" (two customers booking the same room for the same dates) is strictly more important than accepting a request during a partial system failure.

To achieve strong consistency:
\begin{itemize}
    \item \textbf{Distributed Locking with Redis:} Before a transaction is finalized in the persistent database, a temporary lock (Time-To-Live of 15 minutes) is acquired on Redis. This acts as a pessimistic locking mechanism, ensuring that once a user selects a room for payment, no other nodes can write to that specific resource.
    \item \textbf{Atomic Writes in MongoDB:} The final confirmation is performed using MongoDB's atomic document updates. If the database primary node is unreachable (partition), the system rejects the write request rather than allowing a potential state conflict, effectively sacrificing Availability to preserve Data Integrity.
\end{itemize}

\subsection{Search and Discovery: Prioritizing Availability (AP)}
For the browsing, searching, and property listing modules, the system prioritizes \textbf{Availability}. It is acceptable for a user to see a room that was booked seconds ago as "available" in the search results, provided the error is caught at the checkout stage. It is not acceptable, however, for the homepage or search filters to be unresponsive.

To achieve high availability:
\begin{itemize}
    \item \textbf{Caching Strategy:} Frequently accessed data (property details, amenities) are cached in Redis. If the primary database experiences high latency or downtime, the system serves data from the cache.
    \item \textbf{Read Replicas:} The system allows reading from secondary database replicas, accepting that there may be a slight replication lag (stale data) in exchange for system responsiveness.
\end{itemize}

\subsection{Polyglot Persistence and Eventual Consistency}
The system utilizes a polyglot persistence architecture, syncing data between the document store (MongoDB) and the graph database (Neo4j). This synchronization follows an \textbf{Eventual Consistency} model.

When a reservation is confirmed in MongoDB (the "Source of Truth"), the relationship is propagated to Neo4j asynchronously. While there is a minimal delay between the transaction confirmation and the update of the social graph, this does not impact the transactional integrity of the booking. This approach ensures that the Graph Service remains decoupled and does not block the critical path of the user experience.

In conclusion, the system is \textbf{CP} regarding write operations for financial transactions and inventory management, while remaining \textbf{AP} for read-heavy operations such as search and recommendation generation.
\section{Data Distribution and Replication Strategy}
\label{sec:data_distribution}

According to the non-functional requirements outlined in Section 2.1.6, our system must ensure \textbf{High Availability} and \textbf{Partition Tolerance} to guarantee low-latency responses to user requests, particularly for search and browsing operations. 

In the context of the \textbf{CAP theorem}, LargeBeB is designed primarily as an \textbf{AP system} (Availability and Partition Tolerance), accepting \textit{Eventual Consistency} for read operations to maximize throughput. However, to prevent critical issues such as double bookings, we enforce stronger consistency constraints on write operations through specific Write Concerns.

The fundamental aspect of our design relies on a hybrid approach combining \textbf{Replica Sets} for high availability and \textbf{Sharding} for horizontal scalability.

\subsection{Replica Set Configuration}
To ensure redundancy and automatic failover, the MongoDB architecture is deployed as a \textbf{Replica Set} consisting of three nodes. This configuration protects the system against single-node failures and network partitions.

The cluster is composed of one \textbf{Primary} node and two \textbf{Secondary} nodes. To optimize the election process and resource utilization, we have configured specific \textbf{priority levels} for each member:

\begin{figure}[H]
    \centering
    \includegraphics[width=1\textwidth]{images/DBArchitectureRappresentation.png}
    \caption{Database Architecture: Replica Set with 3 Nodes}
    \label{fig:db_architecture}
\end{figure}

\begin{itemize}
    \item \textbf{Node A (IP 10.1.1.21 - Priority 5):} This is the preferred Primary node. It resides on the highest-performance hardware and handles all write operations under normal conditions.
    \item \textbf{Node B (IP 10.1.1.22 - Priority 2):} This node acts as the immediate hot-standby. It serves read operations and is the first candidate to become Primary if Node A fails.
    \item \textbf{Node C (IP 10.1.1.23 - Priority 1):} This node has the lowest priority. It is primarily used for read offloading and acts as a disaster recovery option. Assigning a lower priority ensures it is elected as Primary only as a last resort.
\end{itemize}

\subsection{Sharding Strategy and Partitioning Algorithm}
Given the potential growth of the \texttt{properties} and \texttt{reviews} collections, a single replica set might eventually become a bottleneck. Therefore, we have designed a \textbf{Sharding} strategy to distribute data across multiple shards (horizontal scaling).

\subsubsection{Shard Key Selection}

\paragraph{Properties Collection}
The \texttt{properties} collection requires careful balancing between query performance (mostly geospatial) and write distribution.
\begin{itemize}
    \item \textbf{Shard Key:} \texttt{\{ \_id: "hashed" \}}
    \item \textbf{Rationale:} While a range-based shard key (e.g., based on \texttt{city}) might seem intuitive, it leads to \textit{jumbo chunks} (e.g., "Rome" would be much larger than a small village). Hashed sharding ensures a strictly uniform distribution of properties across the cluster.
    \item \textbf{Note on Rooms:} As defined in our data model, \texttt{Room} entities are embedded within the \texttt{Property} document as a list (\texttt{List<Room>}). Therefore, sharding the parent \texttt{properties} collection automatically scales room availability data without requiring a separate sharding strategy.
\end{itemize}

\paragraph{Reviews Collection}
The \texttt{reviews} collection is an append-only dataset that is expected to grow indefinitely as users rate their stays.
\begin{itemize}
    \item \textbf{Shard Key:} \texttt{\{ \_id: "hashed" \}}
    \item \textbf{Rationale:} To optimize for high write throughput and prevent "hot shards" during peak seasons, we hash the review ID. This ensures that incoming reviews are spread evenly across all shards rather than filling up a single node based on time or property grouping.
\end{itemize}

\paragraph{Users Collection}
The \texttt{users} collection implements a Single Table Inheritance strategy, storing both \texttt{Customer} and \texttt{Manager} entities in the same collection. Currently, this collection is not sharded as the volume of user profiles fits comfortably within the Replica Set capacity, avoiding the overhead of cross-shard queries during login authentication.

\subsection{Read and Write Policies}
To balance the trade-off between latency and data integrity, we have configured specific policies for reading and writing data.

\subsubsection{Read Operations (Read Preference: \texttt{nearest})}
Given that the read-to-write ratio in LargeBeB is estimated to be approximately 10:1 (browsing vs. booking), minimizing read latency is crucial.
We utilize the \textbf{\texttt{nearest}} read preference. The driver routes the read request to the replica set member with the lowest network latency, regardless of whether it is Primary or Secondary.

This choice provides two main benefits:
\begin{enumerate}
    \item \textbf{Load Balancing:} It distributes the heavy read traffic across all three nodes, preventing the Primary from being overwhelmed.
    \item \textbf{Geo-latency Reduction:} If the infrastructure is geo-distributed in the future, users will automatically read from the data center closest to them.
\end{enumerate}

\subsubsection{Write Operations (Write Concern: \texttt{majority})}
Although we prioritize availability for reads, write operations—specifically \textit{User Registration}, \textit{Property Creation}, and critical \textit{Reservation} flows—require strict durability.
We have configured the Write Concern to \textbf{\texttt{majority}}.

\begin{itemize}
    \item The system acknowledges a write only after it has been committed to the Primary and propagated to at least one Secondary (2 out of 3 nodes).
    \item This ensures that even if the Primary node crashes immediately after a write (e.g., after a confirmed payment), the data is safe on a Secondary node that will become the new Primary, preventing data rollback and financial inconsistencies.
\end{itemize}